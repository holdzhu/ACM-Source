\documentclass[a4paper]{ctexart}
	\usepackage{geometry}
	\usepackage{ulem}
	\geometry{left=3.18cm,right=3.18cm,top=2.54cm,bottom=2.54cm}
	\title{Day 1题解报告}
	\author{2014060105005{ }何柱{ }男}
\begin{document}
	\maketitle
	\appendix
	\section{Your Pyramid Starts Here}
	本题无需过多考虑,用字符串数组把0$\sim$9十个数字的Soroban表示出来,然后用循环把n的每一位显示出来,但要注意题目中的"We can assume that number 0 has no leading zeroes.",即0要做特殊处理。
	\section{Bigger Challenge}
	本题直接把所有数字读入一个数组A[N],用排序算法将数组的元素从小到大排列,A[N/2]即是Answer。
	\section{Techniques more subtle}
	典型的BFS题。将整个3D dungeon输入到3维数组A[L][R][C],定义一个由坐标及时间组成的结构体,以及一个由这个结构体组成的队列,确定初始位置后,将初始位置及时间为0入列,进入循环,出列,遍历6个方向,若某个方向可走且没走过(剪枝),将新的坐标及新的时间入列,循环直到队列为空或找到终点,队列为空即“Trapped!”,找到终点即可输出时间。
	\section{Advanced skills}
	本题的数据范围上限为200000,即要求时间复杂度为O(N)。遍历字符串,记录当前字符以及当前字符连续出现的次数,以及上一种字符是否重复了。若上一种字符重复了,则当前字符最多只能输出1个,即当前字符再次出现时不进行操作;若上一种字符不重复,但当前的字符已经重复2次了,也不进行操作;其他情况均输出当前字符。
	\section*{\sout{E{ }{ }{ }Serious challenges}}
	由于本人在比赛当天下午1:30需参加CCF计算机软件能力认证考试,故没时间完成E题,请见谅。
\end{document}