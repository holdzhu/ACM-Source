\documentclass[a4paper]{ctexart}
	\usepackage{geometry}
	\usepackage{enumerate}
	\usepackage{ntheorem}
	\usepackage{tikz}
	\usepackage{slashbox}
	\usepackage{tabularx}
	\usepackage[fleqn]{amsmath}
	\geometry{left=2.44cm,right=2.44cm,top=2.44cm,bottom=2.44cm}
	\title{J - 四句话题意}
	\author{何柱}
	\allowdisplaybreaks
\begin{document}
	\maketitle
	对于每个$n$的结果分解质因数。令$f(n,p)$表示$n-1$对应的结果的质因数中含有质数$p$的个数,打表如下
	\newcolumntype{C}{>{\centering\arraybackslash} m{0.018\linewidth} }
	\begin{figure}[h]
		\begin{center}
			\begin{tabular}{|c|C|C|C|C|C|C|C|C|C|C|C|C|C|C|C|C|C|C|C|C|}\hline
				\backslashbox{p\kern-1em}{\kern-1em n} & 1 & 2 & 3 & 4 & 5 & 6 & 7 & 8 & 9 & 10 & 11 & 12 & 13 & 14 & 15 & 16 & 17 & 18 & 19 & 20 \\ \hline
				2 & 0 & 0 & 1 & 0 & 2 & 1 & 2 & 0 & 3 & 2 & 3 & 1 & 3 & 2 & 3 & 0 & 4 & 3 & 4 & 2 \\ \hline
				3 & 0 & 0 & 0 & 1 & 1 & 0 & 1 & 1 & 0 & 2 & 2 & 1 & 2 & 2 & 1 & 2 & 2 & 0 & 2 & 2 \\ \hline
				5 & 0 & 0 & 0 & 0 & 0 & 1 & 1 & 1 & 1 & 0 & 1 & 1 & 1 & 1 & 0 & 1 & 1 & 1 & 1 & 0 \\ \hline
			\end{tabular}
		\end{center}
		\caption{$f(n,p)$}
	\end{figure}

	归纳可以得出
	$$f(n,p)=\begin{cases}
	f(\frac{n}{p},p) & n \bmod p = 0 \\ 
	\lfloor\log_pn\rfloor & n \bmod p \neq 0
	\end{cases}$$

	于是,答案就出来了
	$$ans(n)=\prod_{p \in primes,p\leq n}p^{f(n+1,p)}$$
	其中幂运算可以用快速幂算法。
\end{document}