\documentclass[a4paper]{ctexart}
	\usepackage{geometry}
	\usepackage{enumerate}
	\usepackage{ntheorem}
	\usepackage{tikz}
	\usepackage{slashbox}
	\usepackage{tabularx}
	\usepackage[fleqn]{amsmath}
	\geometry{left=3.18cm,right=3.18cm,top=2.44cm,bottom=2.44cm}
	\title{G - 三句话题意}
	\author{何柱}
	\allowdisplaybreaks
	\setcounter{MaxMatrixCols}{20}
	\newcommand{\block}[1]{
		\underbrace{\begin{matrix}1 & \cdots & 1\end{matrix}}_{#1}
	}
\begin{document}
	\maketitle
	每次操作后的所有数字为操作前所有数字的线性组合。设操作了t次后的数字组成的列向量为$v_t$,则有以下关系:
	$$v_{t+1}=Av_t$$
	其中
	$$A=\begin{pmatrix}
		1 & \block{d} &&&&& \block{d} \\
		1 & 1 & \block{d} &&&& \block{d - 1} \\
		1 & 1 & 1 & \block{d} &&& \block{d - 2} \\
		\vdots & \vdots & \vdots & \vdots &&& \vdots \\
		\block{d} &&&&&& \block{d + 1}
	\end{pmatrix}$$
	可以推出
	$$v_{t}=A^kv_0$$
	其中$A^k$可以用快速幂算法来做。

	此外,因为$A$为循环矩阵,所以对于乘法封闭,即两个循环矩阵的乘积依然是循环矩阵,所以对于循环矩阵只需要记录第一行,乘法运算中计算出第一行即可。
\end{document}