\documentclass[a4paper]{ctexart}
	\usepackage{geometry}
	\usepackage{enumerate}
	\usepackage{ntheorem}
	\usepackage{tikz}
	\usepackage{slashbox}
	\usepackage{tabularx}
	\usepackage[fleqn]{amsmath}
	\geometry{left=3.18cm,right=3.18cm,top=2.44cm,bottom=2.44cm}
	\title{F - 两句话题意}
	\author{何柱}
	\allowdisplaybreaks
\begin{document}
	\maketitle
	先说一个比较重要的定理,设$f(n)$和$g(n)$为任意函数,下面两个命题等价:
	\begin{enumerate}
		\item[(i)] $$g(n)=\sum_{m=1}^{\infty}f(mn),n=1,2,3,\ldots$$
		且
		$$\exists\varepsilon>0,\sum_{n=1}^{\infty}n^\varepsilon|f(n)|\text{收敛}$$
		\item[(ii)] $$f(n)=\sum_{m=1}^{\infty}\mu(m)g(mn),n=1,2,3,\ldots$$
		且
		$$\exists\varepsilon>0,\sum_{n=1}^{\infty}n^\varepsilon|g(n)|\text{收敛}$$
	\end{enumerate}
	其中$\mu$为默比乌斯函数
	$$\mu(n)=\begin{cases}
	1 & \text{若}n=1 \\
	(-1)^k & \text{若}n\text{无平方数因数且}n=p_1p_2\ldots p_k \\
	0 & \text{若}n\text{有大于}1\text{的平方数因数}
	\end{cases}$$
	令$f(n)$表示最大公约数为$n$的组数,$g(n)$表示有公约数$n$的组数,显然有
	$$g(n)=\sum_{m=1}^{\infty}f(mn),n=1,2,3,\ldots,a_{max}$$
	根据定理有
	$$f(n)=\sum_{m=1}^{\infty}\mu(m)g(mn),n=1,2,3,\ldots,a_{max}$$
	令$c(n)$表示集合中含有因子$n$的数字个数,则
	$$g(n)=2^{c(n)}-1,n=1,2,3,\ldots,a_{max}$$
	当$mn$超过集合元素的最大值$a_{max}$时,对$f(n)$的贡献永远为$0$,不必计算,即
	$$f(n)=\sum_{m=1}^{\lfloor\frac{a_{max}}{n}\rfloor}\mu(m)g(mn),n=1,2,3,\ldots,a_{max}$$
	根据题意,期望为
	$$E=\frac{\sum_{x=1}^{a_{max}}x^kf(x)}{2^n-1}$$
	要求的答案为
	$$ans=(2^n-1)E \bmod 10000007$$
	整理得
	$$ans=(\sum_{x=1}^{a_{max}}x^k\sum_{m=1}^{\lfloor\frac{a_{max}}{x}\rfloor}\mu(m)g(mx)) \bmod 10000007$$
	其中
	$$g(n)=2^{c(n)}-1,n=1,2,3,\ldots,a_{max}$$
	预处理函数$g(n)$的值,计算$x^k$时使用快速幂算法,时间复杂度为$O(n+a_{max}loga_{max}+a_{max}logk)$。
\end{document}