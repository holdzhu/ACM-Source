\documentclass[a4paper]{ctexart}
	\usepackage{geometry}
	\usepackage{enumerate}
	\usepackage{ntheorem}
	\usepackage{tikz}
	\usepackage{slashbox}
	\usepackage{tabularx}
	\usepackage[fleqn]{amsmath}
	\geometry{left=2.44cm,right=2.44cm,top=2.44cm,bottom=2.44cm}
	\title{E - 王老板和仓鼠}
	\author{何柱}
	\allowdisplaybreaks
\begin{document}
	\maketitle
	在操作3中,可以证明,最后一定会把$b$变成$b\bmod a$,整个操作与求$gcd$十分相似。假设操作者遇到状态$(a,b)$,不妨设$a\leq b$,如果$(a,b\bmod a)$是必输状态,那么他可以毫不犹豫地把$b$换成$b\bmod a$,即$(a,b)$为必赢状态;但是如果$(a,b\bmod a)$是必赢状态,是否存在一个必输状态$(a,b-a^k)$呢?在这个过程中,状态只与两个值$b-b\bmod a$和$a$有关。如果我们能求出状态和这两个值的关系,设为$f(x,y)$,就可以知道$(a,b)$的状态。通过打表可以归纳出$$f(x,y)=\lfloor\frac{x-1}{y+1}\rfloor\bmod(y+1)\bmod 2$$

	所以当$(a,b\bmod a)$是必输状态或者$f(b-b\bmod a,a)=0$且$(a,b\bmod a)$是必赢状态时,$(a,b)$为必赢状态;否则$(a,b)$为必输状态。终止状态$(0,x)$为必输状态。
	\newcolumntype{C}{>{\centering\arraybackslash} m{0.018\linewidth} }
	\begin{figure}[h]
		\begin{center}
			\begin{tabular}{|c|C|C|C|C|C|C|C|C|C|C|C|C|C|C|C|C|C|C|C|C|}\hline
				\backslashbox{y\kern-1em}{\kern-1em x} & 1 & 2 & 3 & 4 & 5 & 6 & 7 & 8 & 9 & 10 & 11 & 12 & 13 & 14 & 15 & 16 & 17 & 18 & 19 & 20 \\ \hline
				1 & 0 & 1 & 0 & 1 & 0 & 1 & 0 & 1 & 0 & 1 & 0 & 1 & 0 & 1 & 0 & 1 & 0 & 1 & 0 & 1 \\ \hline
				2 & 0 & 0 & 1 & 1 & 1 & 1 & 0 & 0 & 1 & 1 & 1 & 1 & 0 & 0 & 1 & 1 & 1 & 1 & 0 & 0 \\ \hline
				3 & 0 & 0 & 0 & 1 & 1 & 1 & 0 & 0 & 0 & 1 & 1 & 1 & 0 & 0 & 0 & 1 & 1 & 1 & 0 & 0 \\ \hline
				4 & 0 & 0 & 0 & 0 & 1 & 1 & 1 & 1 & 0 & 0 & 0 & 0 & 1 & 1 & 1 & 1 & 1 & 1 & 1 & 1 \\ \hline
				5 & 0 & 0 & 0 & 0 & 0 & 1 & 1 & 1 & 1 & 1 & 0 & 0 & 0 & 0 & 0 & 1 & 1 & 1 & 1 & 1 \\ \hline
				6 & 0 & 0 & 0 & 0 & 0 & 0 & 1 & 1 & 1 & 1 & 1 & 1 & 0 & 0 & 0 & 0 & 0 & 0 & 1 & 1 \\ \hline
			\end{tabular}
		\end{center}
		\caption{f(x,y)}
	\end{figure}
\end{document}